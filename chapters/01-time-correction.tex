% ============================================
% الفصل الأول: تصحيح الزمن – برهان قانون الحركة
% ============================================
\chapter{تصحيح الزمن: برهان قانون الحركة}
\label{chap:time-correction}

\section{مقدمة: المؤثر والنتيجة}
يعتمد هذا البحث على قانون الحركة الكونية (المؤثر والنتيجة). في هذا الفصل، نثبت أن العجز التراكمي في ساعات شروق القمر (10 ساعات) هو المؤثر الذي يقود للثابت السيادي.

\section{العجز التراكمي: \deficitHours\ ساعات}
تم رصد العجز الكوني التراكمي في إجمالي ساعات شروق القمر كقيمة ثابتة:
\begin{equation}
\Delta \tau = \deficitHours \text{ hours}
\end{equation}

\section{التحقق من الثابت الموحد \unified}
باستخدام قانون الحركة، نربط بين العجز (10 ساعات) والقيمة النسبية (43) ومعامل التربيع (800):

\begin{sovereignbox}{البرهان الرياضي للمؤثر والنتيجة}
\[ 800 \times 43 = 34,400 \]
وبالربط مع العجز التراكمي (10 ساعات):
\begin{align*}
\text{Total Deficit} &= (800 \times 4.3) \times \deficitHours \\
&= 3440 \times 10 \\
&= \unified
\end{align*}
\end{sovereignbox}

\section{الخلاصة الجيوديسية}
هذا التوافق الرقمي التام بين العجز الزمني (\deficitHours\ ساعات) والثابت (\unified) ليس صدفة، بل هو الضابط لمحيط الأرض الجيوديسي المقدر بـ \geodesic\ كم، وهو ما يثبت دقة ميزان الملكوت.
