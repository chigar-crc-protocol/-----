% ============================================
% الفصل الخامس: البينة الجيوديسية - الأرض تشهد
% ============================================
\chapter{البينة الجيوديسية: قرن الأرقام بالواقع}

\section{التطابق الرقمي الكوني}
في هذا الفصل، ننتقل من المعادلات النظرية إلى الشواهد الملموسة على أرض الواقع، حيث نربط بين الثابت الهندسي ومحيط الأرض الجيوديسي.

\section{ثابت الـ 432 ومحيط الأرض}
نلاحظ التطابق المذهل بين الثابت الهندسي الذي وثقناه في الفصل الثاني وبين القياسات الأرضية الحقيقية:

\begin{sovereignbox}{البرهان الجيوديسي}
بما أن محيط الأرض الجيوديسي المعتمد هو \textbf{43,200 كم}، وبما أن الثابت السيادي لمركزية مكة مرتبط بالرقم \textbf{432}، فإن العلاقة السيادية هي:
\[ \text{Earth Circumference} = 432 \times 100 \]
هذا التطابق يثبت أن إحداثيات المركز (مكة) ليست صدفة، بل هي ضابطة لقياس الكوكب بالكامل.
\end{sovereignbox}

\section{الارتباط بقانون الحركة}
سنقوم هنا بدمج نتائج الفصل الثالث (قانون الحركة التكعيبي) لإثبات كيف تؤثر التغيرات الزمنية الطفيفة على دوران الأرض وفقاً للمعادلة:
\[ \mathcal{E} = 34.4 \times \mathcal{C}^3 \]
\end{sovereignbox}

\section{الخلاصة الرصدية}
إن اتساق هذا النموذج مع القياسات الحديثة يؤكد أن "ميزان الملكوت" هو المفتاح لفك شفرة الهندسة الكونية للأرض.
