% ============================================
% الفصل الثالث: قانون الحركة – السبب والنتيجة
% ============================================
\chapter{قانون الحركة: السبب والنتيجة}

\section{المبدأ الكوني للسببيّة}
يقوم الكون على مبدأ أساسي: لكل سبب نتيجة. في هذا الفصل، نوثق القانون الحركي الذي يربط بين الثوابت السيادية، ويشرح كيف ينشأ "التصحيح الزمني".

\section{تعريف القانون الحركي}
نصيغ العلاقة بين السبب (المؤثر) والنتيجة (الأثر) وفق المعادلة التكعيبية التالية:
\begin{sovereignbox}{معادلة الحركة الكونية}
إذا اعتبرنا العجز الزمني (10 ساعات) هو السبب $\mathcal{C}$، فإن النتيجة $\mathcal{E}$ هي:
\[ \mathcal{E} = 34.4 \times \mathcal{C}^3 \]
بما أن $\mathcal{C} = 10$ ساعات، فإن:
\[ \mathcal{E} = 34.4 \times 1000 = 34400 \]
وهو يتطابق تماماً مع الثابت الموحد \textbf{34400}.
\end{sovereignbox}

\section{تفسير الارتباط الزمني}
هذا القانون يثبت أن التراكم البسيط في ساعات العجز الكوني يؤدي إلى نتائج هندسية كبرى، مما يربط الفصل الأول (الزمن) بالفصول القادمة (سقر والجيوديسيا).
