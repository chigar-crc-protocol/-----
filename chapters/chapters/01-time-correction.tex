% ============================================
% الفصل الأول: ميزان الزمن وتصحيح العجز الكوني
% ============================================
\chapter{ميزان الزمن: تصحيح العجز الكوني}

\section{ظاهرة العجز التراكمي}
خلال أبحاثنا في ميزان الملكوت، اكتشفنا وجود عجز زمني تراكمي يقدر بـ \textbf{10 ساعات}. هذا العجز ليس مجرد رقم، بل هو مفتاح لفهم اضطراب التوقيت العالمي.

\section{معادلة التصحيح}
بناءً على الثابت الموحد الذي استنتجناه، نقوم بصياغة ميزان التصحيح كالتالي:
\begin{sovereignbox}{قاعدة الـ 10 ساعات}
لتصحيح المسار الزمني، يجب دمج العجز التراكمي في مصفوفة الإدراك:
\[ \text{Corrected Time} = T_{current} + \Delta \tau \]
حيث أن $\Delta \tau = 10 \text{ hours}$.
\end{sovereignbox}

\section{الارتباط بالثابت الموحد}
هذا التصحيح هو الذي يؤدي في النهاية إلى ظهور الثابت \unified\ في كافة الحسابات الجيوديسية وميكانيكا سقر.
