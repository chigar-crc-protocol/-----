\documentclass[12pt,a4paper,oneside]{book}
% ========== preamble.tex – النسخة السيادية الموحدة ==========

% 1. الحزم الأساسية للغة والتنسيق
\usepackage[utf8]{inputenc}
\usepackage[LAE,LFE]{fontenc}
\usepackage[arabic,english]{babel}
\usepackage{amsmath, amssymb, amsthm} % للحسابات والمعادلات
\usepackage{geometry}
\geometry{a4paper, margin=2.5cm}
\usepackage{xcolor}
\usepackage{tcolorbox} % للصناديق الملونة

% 2. تعريف الألوان السيادية
\definecolor{SovereignBlue}{RGB}{0, 51, 102}
\definecolor{SovereignGold}{RGB}{212, 175, 55}

% 3. الثوابت السيادية (تعريف برمجى لاستخدامها في المعادلات)
\newcommand{\unified}{34400}          % الثابت الموحد
\newcommand{\geodesic}{43200}         % محيط الأرض الجيوديسي (كم)
\newcommand{\saqarOrbit}{51840000000} % مدار سقر السنوي (كم)
\newcommand{\lightHours}{48}          % مكافئ سقر بالساعات الضوئية
\newcommand{\deficitHours}{10}        % العجز التراكمي (ساعات)

% 4. تنسيق الصناديق (Sovereign Box)
\newtcolorbox{sovereignbox}[1]{
  colframe=SovereignBlue,
  colback=white,
  fonttitle=\bfseries,
  title=#1,
  arc=5pt,
  outer arc=5pt
}

% 5. أمر للكتابة بالعربية والإنجليزية معاً
\newcommand{\bilingual}[2]{
  \selectlanguage{arabic}#1
  \selectlanguage{english}\par\textit{#2}
  \selectlanguage{arabic}
}

% نهاية ملف preamble


\begin{document}

% 1. الغلاف السيادي
\begin{titlepage}
\centering
\vspace*{1.5cm}

\begin{tcolorbox}[colframe=SovereignBlue, colback=white, arc=20pt, linewidth=5pt, center, width=0.9\textwidth]
    \centering
    \vspace{0.8cm}
    {\Huge\bfseries\color{SovereignBlue} المجلد الكوني اليقين}\\[0.5cm]
    {\Large\bfseries مصفوفة الإدراك الهندسي والزمني}\\[0.3cm]
    {\small الإصدار السيادي الموحد — 2026}
    \vspace{0.8cm}
\end{tcolorbox}

\vspace{2cm}

\begin{tikzpicture}
    \draw[line width=5pt, SovereignBlue] (0,0) circle (3.5);
    \node at (0,0.3) {\fontsize{55}{65}\selectfont\bfseries 34400};
    \node at (0,-1.2) {\Large\bfseries القيمة السيادية الموحدة};
\end{tikzpicture}

\vfill

{\Huge\bfseries الباحث: هشام شيكر}\\[0.4cm]
{\large المرجعية العلمية: بيانات الإمام المهدي ناصر محمد اليماني}\\[0.8cm]

\hrule
\vspace{0.4cm}
{\small تم التوثيق الرقمي في: 13 فبراير 2026 م}
\end{titlepage}

% 2. الفهرس التلقائي
\newpage
\tableofcontents 
\newpage

% 3. مقدمة اليقين
\chapter*{مقدمة اليقين}
\addcontentsline{toc}{chapter}{مقدمة اليقين}
بسم الله ميزان الحق والعدل. نضع بين يدي العالم هذا "المجلد الكوني اليقين"، لتوثيق الثوابت السيادية وعلى رأسها الثابت الموحد \textbf{34400}، ونسبة الإضاءة الكونية \textbf{432/55}، ومركزية مكة المكرمة كقطب لليابسة.

% 4. الفصل الأول
\chapter{مركزية بكة ومصفوفة الإدراك}
\section{البرهان الإحداثي}
إن مكة المكرمة هي المركز المطلق لليابسة هندسياً، وهذا البحث يعيد تعريف إحداثيات الكوكب بناءً على هذا المركز السيادي.

% 5. الفصل الثاني
\chapter{الثوابت الكونية وقانون الكدح}
\section{العجز الكوني}
نوثق هنا العجز التراكمي في ساعات شروق القمر البالغ 10 ساعات، وعلاقته بالثابت 800 و 43 لإنتاج القيمة السيادية 34400.

\end{document}
